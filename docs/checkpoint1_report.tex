\documentclass[12pt,a4paper]{article}
\usepackage[utf8]{inputenc}
\usepackage[T2A]{fontenc}
\usepackage[russian]{babel}
\usepackage{graphicx}
\usepackage{hyperref}
\usepackage{geometry}
\usepackage{float}

\geometry{margin=2cm}
\hypersetup{colorlinks=true, linkcolor=blue, urlcolor=blue}
\setlength{\parindent}{0pt}
\setlength{\parskip}{0.5em}

\begin{document}

\begin{center}
\Large\textbf{Чекпоинт 1: Сбор и подготовка данных}\\[0.3em]
\large RAG Чат-бот по роману «Мастер и Маргарита»\\[1em]
\normalsize\textbf{Команда bookworm}\\[0.3em]
Горбунов Дмитрий Павлович\\
Ковалева Дарина Евгеньевна\\
Тельнов Федор Николаевич\\
Мацаков Борис Вячеславович
\end{center}

\vspace{0.5em}
\textbf{Репозиторий:} \url{https://github.com/bookworm-itmo/llm-intro-project}

\section*{Источник данных}

Роман М.А. Булгакова «Мастер и Маргарита» в формате FB2, скачан из электронной библиотеки \href{https://flibusta.is/b/813954}{Флибуста}. Книга подходит для RAG: большой объём текста, много персонажей и событий.

\section*{Сбор данных}

Разработан парсер FB2, который:
\begin{itemize}
    \item Извлекает текст из XML-структуры файла
    \item Разбивает на главы по паттерну «Глава N»
    \item Сохраняет текст с привязкой к номерам глав
\end{itemize}

Исходный файл: \href{https://github.com/bookworm-itmo/llm-intro-project/blob/main/data/master_and_margarita.fb2}{\texttt{data/master\_and\_margarita.fb2}}

\section*{Подготовка для RAG}

\textbf{Chunking} — разбиение текста на фрагменты с помощью \texttt{RecursiveCharacterTextSplitter}:
\begin{itemize}
    \item Размер чанка: 800 символов, перекрытие: 100 символов
    \item Каждому чанку присвоен номер главы
    \item Результат: \textbf{1182 чанка} из 32 глав
\end{itemize}

\textbf{Эмбеддинги} — векторизация через GigaChat Embeddings (Сбер):
\begin{itemize}
    \item Размерность: 1024, нормализация L2
\end{itemize}

\textbf{Индексация} — FAISS IndexFlatIP для косинусного поиска.

\section*{Структура данных}

Все данные в папке \href{https://github.com/bookworm-itmo/llm-intro-project/tree/main/data}{\texttt{data/}}:

\begin{verbatim}
data/
├── master_and_margarita.fb2   # Исходная книга
├── chunks.parquet             # Чанки (chunk_id, chapter, text)
├── embeddings.parquet         # Векторы (chunk_id, embedding[1024])
├── chunks_sample.json         # Сэмпл данных
└── faiss_index/index.faiss    # Векторный индекс
\end{verbatim}

\section*{Объём данных}

\begin{tabular}{ll}
Исходный текст: & $\sim$580 000 символов \\
Глав: & 32 + эпилог \\
Чанков: & 1 182 \\
Средний размер чанка: & $\sim$705 символов \\
\end{tabular}

\section*{Сэмпл данных}

Доступен по ссылке: \href{https://github.com/bookworm-itmo/llm-intro-project/blob/main/data/chunks_sample.json}{\texttt{data/chunks\_sample.json}}

Пример чанка (глава 1):
\begin{quote}
\small
«Никогда не разговаривайте с неизвестными. В час жаркого весеннего заката на Патриарших прудах появилось двое граждан. Первый из них — приблизительно сорокалетний, одетый в серенькую летнюю пару, — был маленького роста, темноволос, упитан, лыс...»
\end{quote}

\section*{Архитектура}

Система состоит из трёх сервисов:
\begin{itemize}
    \item \textbf{Data Service} — парсинг FB2, chunking, подготовка данных
    \item \textbf{RAG Service} — поиск релевантных фрагментов через FAISS
    \item \textbf{LLM Service} — генерация ответов через Claude API
\end{itemize}

Пользователь взаимодействует с системой через веб-интерфейс (Streamlit).

\begin{figure}[H]
\centering
\includegraphics[width=\textwidth]{solution_architecture.png}
\caption{Архитектура RAG чат-бота}
\end{figure}

\end{document}
