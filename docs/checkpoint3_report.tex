\documentclass[12pt,a4paper]{article}
\usepackage[utf8]{inputenc}
\usepackage[T2A]{fontenc}
\usepackage[russian]{babel}
\usepackage{graphicx}
\usepackage{hyperref}
\usepackage{geometry}
\usepackage{float}
\usepackage{booktabs}
\usepackage{longtable}

\geometry{margin=2cm}
\hypersetup{colorlinks=true, linkcolor=blue, urlcolor=blue}
\setlength{\parindent}{0pt}
\setlength{\parskip}{0.5em}

\begin{document}

\begin{center}
\Large\textbf{Чекпоинт 3: Деплой и интерфейс}\\[0.3em]
\large RAG Чат-бот по роману <<Мастер и Маргарита>>\\[1em]
\normalsize\textbf{Команда bookworm}\\[0.3em]
Горбунов Дмитрий Павлович\\
Ковалева Дарина Евгеньевна\\
Тельнов Федор Николаевич\\
Мацаков Борис Вячеславович
\end{center}

\vspace{0.5em}
\textbf{Репозиторий:} \url{https://github.com/bookworm-itmo/llm-intro-project}

\section*{1. Функционал интерфейса}

Веб-интерфейс реализован на \textbf{Streamlit} и представляет собой чат-бота для ответов на вопросы по роману М.А. Булгакова <<Мастер и Маргарита>>.

\subsection*{Основные возможности}

\begin{tabular}{ll}
\toprule
\textbf{Функция} & \textbf{Описание} \\
\midrule
Чат-интерфейс & Классический формат диалога с историей сообщений \\
RAG-поиск & Поиск релевантных фрагментов через FAISS + GigaChat Embeddings \\
Генерация ответов & OpenRouter (GPT-4o-mini по умолчанию) или Claude Haiku \\
Источники & Раскрывающийся блок с цитатами из книги под каждым ответом \\
Реранкер & Переключатель в сайдбаре для BGE Reranker \\
\bottomrule
\end{tabular}

\subsection*{Элементы интерфейса}

\begin{enumerate}
    \item \textbf{Заголовок} --- <<Чат-бот по роману `Мастер и Маргарита'>> с пояснением
    \item \textbf{Сайдбар} --- чекбокс <<Использовать реранкер>> (включен по умолчанию)
    \item \textbf{История чата} --- все предыдущие вопросы и ответы сессии
    \item \textbf{Поле ввода} --- внизу экрана, placeholder <<Задайте вопрос о книге>>
    \item \textbf{Спиннер} --- индикатор загрузки <<Ищу ответ в книге...>>
    \item \textbf{Экспандер источников} --- <<Источники из книги>> с номером главы и текстом
\end{enumerate}

\subsection*{Обработка ошибок}

\begin{itemize}
    \item При отсутствии API-ключей --- ошибка при запуске
    \item При сбое LLM/embedding API --- исключение отображается в интерфейсе
    \item Rate limit --- обрабатывается с retry и задержками
\end{itemize}

\section*{2. Скриншот интерфейса}

\begin{figure}[H]
\centering
\includegraphics[width=0.95\textwidth]{main_screen.png}
\caption{Интерфейс чат-бота: вопрос <<Что случилось с Берлиозом?>>, ответ с источниками, развернутый блок цитат из глав. Слева --- сайдбар с переключателем реранкера.}
\end{figure}

\section*{3. Инструкция по запуску}

\subsection*{Требования}

\begin{itemize}
    \item Python 3.10+
    \item API-ключ OpenRouter (по умолчанию) или Claude (Anthropic)
    \item API-ключ GigaChat (Сбер) для эмбеддингов
\end{itemize}

\subsection*{Быстрый старт}

\begin{verbatim}
# Клонировать репозиторий
git clone https://github.com/bookworm-itmo/llm-intro-project
cd llm-intro-project

# Установить зависимости
pip install -r requirements.txt

# Создать .env файл (см. .env.example)
cp .env.example .env
# Заполнить OPENROUTER_API_KEY и GIGACHAT_AUTH_KEY

# Подготовить данные (если нет готовых)
python main.py

# Запустить интерфейс
streamlit run frontend/app.py
\end{verbatim}

\subsection*{Запуск через Docker}

\begin{verbatim}
docker-compose up frontend
\end{verbatim}

\subsection*{Структура .env}

\begin{verbatim}
LLM_PROVIDER=openrouter          # или "claude"
OPENROUTER_API_KEY=sk-or-v1-...  # для OpenRouter
OPENROUTER_MODEL=openai/gpt-4o-mini
CLAUDE_API_KEY=sk-ant-...        # для Claude
GIGACHAT_AUTH_KEY=base64_key     # обязательно
\end{verbatim}

\section*{4. Финальные метрики качества системы}

\subsection*{Метрики RAGAS}

\begin{table}[H]
\centering
\begin{tabular}{lccc}
\toprule
\textbf{Метрика} & \textbf{Без реранкера} & \textbf{С реранкером} & \textbf{Δ} \\
\midrule
Faithfulness & 0.856 & 0.821 & -0.035 \\
Answer Relevancy & 0.358 & 0.398 & +0.040 \\
Context Precision & 0.487 & 0.474 & -0.013 \\
Context Recall & 0.431 & 0.421 & -0.010 \\
\bottomrule
\end{tabular}
\caption{Метрики RAGAS на валидационной выборке (70 запросов, GPT-4o-mini)}
\end{table}

\section*{5. Проведенные эксперименты}

\subsection*{Реранкер (BGE-reranker-base)}

Реранкер улучшил только answer\_relevancy (+0.040), остальные метрики немного просели. Вероятная причина: реранкер переупорядочивает контекст, но для данного датасета это не даёт преимущества — GigaChat эмбеддинги уже хорошо находят релевантные фрагменты.

\subsection*{Что не работало}

\begin{itemize}
    \item \textbf{OpenAI Embeddings} --- проблемы с VPN
    \item \textbf{GigaChat для LLM} --- rate limits (429)
\end{itemize}

\section*{6. Архитектура системы}

\begin{figure}[H]
\centering
\includegraphics[width=\textwidth]{architecture.png}
\caption{C4-диаграмма архитектуры RAG чат-бота (PlantUML)}
\end{figure}

\section*{7. Структура репозитория}

\begin{verbatim}
llm-intro-project/
├── data/                    # Данные (chunks, embeddings, index)
├── services/
│   ├── data_service/        # Парсинг FB2, chunking
│   ├── rag_service/         # FAISS, embeddings, reranker
│   └── llm_service/         # OpenRouter / Claude API
├── frontend/                # Streamlit UI
├── validation/              # Скрипты оценки
├── metrics/                 # Результаты экспериментов
├── docs/                    # Документация
├── requirements.txt         # Зависимости (точные версии)
└── README.md               # Инструкция по запуску
\end{verbatim}

\section*{Заключение}

Реализован RAG чат-бот с веб-интерфейсом для ответов на вопросы по роману <<Мастер и Маргарита>>. Система использует:

\begin{itemize}
    \item \textbf{GigaChat Embeddings} для векторизации текста (1024 dim)
    \item \textbf{FAISS IndexFlatIP} для поиска
    \item \textbf{BGE Reranker} для переранжирования
    \item \textbf{OpenRouter (GPT-4o-mini)} или Claude для генерации ответов
    \item \textbf{Streamlit} для веб-интерфейса
\end{itemize}

Система может быть развернута локально через pip или Docker.

\end{document}
